%---------------------------------------------------------------------------------------------------
%
%	Author: Samuel Acuña
%   Description: Curriculum Vitae
%

%---------------------------------------------------------------------------------------------------
%   DOCUMENT DEFINITION
% 
% we use article class because we want to fully customize the page and dont use a cv template
\documentclass[letterpaper, 10pt]{article}

%---------------------------------------------------------------------------------------------------
%   PACKAGES
%

%\usepackage[utf8]{inputenc} % Required for inputting international characters
%\usepackage[T1]{fontenc} % Output font encoding for international characters

\usepackage[colorlinks=true,linkcolor=black,urlcolor=urlblue]{hyperref} % hyperlinks

\usepackage[top=1in, bottom=1in, left=1in, right=1in]{geometry}   % set margins

\usepackage{datetime} % to create custom date formatting
\newdateformat{monthyeardate}{\monthname[\THEMONTH] \THEYEAR} % new date format: month year

\usepackage{lastpage} % Required for calculating the number of pages in the document

\usepackage{fancyhdr}   % to create custom headers
\pagestyle{fancy}   % use fancy header
\fancyfoot[C]{} % clear default page number
% v1:
%\fancyhead[R]{\small \textsl{Updated: \monthyeardate\today}} % right header
%\fancyfoot[R]{\small \textsl{Acuña SA, CV Page \thepage\ of \pageref{LastPage}}} % right footer
% v2:
\fancyhead[L]{\small \monthyeardate\today} % right header
\fancyhead[R]{\small \thepage\ of \pageref{LastPage}} % right header

\renewcommand{\headrulewidth}{0pt} % remove top header line
\headsep = 1cm %5pt % less space between header and content


%\usepackage{showframe} % for debugging the page, to show outer frames

\setlength{\parindent}{0in} % paragraph indentation is zero

\usepackage{hang}

\setcounter{secnumdepth}{0} % suppress section numbering

\usepackage{enumitem} % to adjust lists formatting
\setlist[itemize]{noitemsep} % reduce spacing for itemized lists

\usepackage{color} % Required for custom colors
\definecolor{urlblue}{RGB}{46,117,182}

\usepackage{titlesec}
\titleformat{\section}{\large\bfseries}{\thesection}{1em}{}
[{\titlerule[0.5pt]}]

\titleformat{\subsection}{\normalsize}{\thesection}{}{}[]

\usepackage{fontspec} % Required for specification of custom fonts
%\setmainfont{Tahoma}
\setmainfont{Optima}





%---------------------------------------------------------------------------------------------------
% BEGIN DOCUMENT
%
\begin{document}

%---------------------------------------------------------------------------------------------------
% TITLE and contact info
% 

{\LARGE \textbf{Samuel A. Acuña}}
{\small \hspace{8pt}  (he/him, \href{https://namedrop.io/samuelacuna}{pronounce})}

\begin{tabbing}{l l}
     \hspace{1cm} \= \hspace{10cm} \kill % Spacing within the block
     Department of Bioengineering\\
     Center for Adaptive Systems of Brain-Body Interactions\\
     George Mason University\\
     Office: \> Rm 3100A Peterson Hall, 4400 University Drive, Fairfax, VA 22030\\
     Email: \> \href{mailto:sacuna2@gmu.edu}{sacuna2@gmu.edu}
     % Twitter: \> \href{https://twitter.com/SamuelAcunaPhD}{@SamuelAcunaPhD}\\
     % LinkedIn: \> \href{https://www.linkedin.com/in/samuelacunaphd}{samuelacunaphd}\\
\end{tabbing}


%---------------------------------------------------------------------------------------------------
%\section{Summary}
%Biomechanical engineer with 10+ years research experience involving rehabilitation engineering, biomechanics, and human-centered product design. I develop new medical technologies to address movement disorders that develop after injury, such as stroke or traumatic brain injury. I'm particularly interested in solving engineering problems for the hospital \& health care industry with innovative and emerging technology, such as virtual reality, ultrasound, and noninvasive neural stimulation. 
%---------------------------------------------------------------------------------------------------
%\section{Research Statment}
% The goal of my research is to... (one sentence)

%---------------------------------------------------------------------------------------------------
\section{Education}

\textbf{Ph.D., Mechanical Engineering}
\hfill 2015 -- 2019\\ % May 2019
University of Wisconsin–Madison %, Madison, WI
\medskip

\textbf{M.S., Mechanical Engineering}
\hfill 2013 -- 2015\\ % May 2015
University of Wisconsin–Madison %, Madison, WI
\medskip

\textbf{B.S., Mechanical Engineering}
\hfill 2005 -- 2012\\ % December 2012
Brigham Young University %, Provo, UT

%---------------------------------------------------------------------------------------------------
\section{Positions}

\textbf{Research Assistant Professor}
\hfill 2023 -- Present\\ % 05/2021 -- Present
George Mason University, Dept. of Bioengineering\\
Center for Adaptive Systems of Brain-Body Interactions
\medskip

\textbf{Adjunct Teaching Faculty}
\hfill 2022 -- Present\\ % 08/2022 --
George Mason University, Dept. of Bioengineering
\medskip

\textbf{Postdoctoral Research Fellow}
\hfill 2021 -- 2023\\ % 05/2021 -- 03/2023
George Mason University, Dept. of Bioengineering
% PI: Siddhartha Sikdar, Ph.D. (Biomedical Imaging Laboratory)
\medskip

\textbf{Patent Technical Advisor}
\hfill 2020 -- 2021\\ % 08/2020 -- 04/2021
Cooley LLP, Patent Prosecution \& Counseling % Reston, VA
\medskip

\textbf{Technical Manager}
\hfill 2019 -- 2020\\ % 12/2019 -- 07/2020
University of Texas at Dallas, UTDesign Senior Engineering Capstone Program %Richardson, TX
\medskip

\textbf{Postdoctoral Researcher}
\hfill 2019 -- 2020\\ % 06/2019 -- 07/2020
University of Texas Southwestern Medical Center, Dept. of Physical Medicine \& Rehabilitation %, Dallas, TX
%PI: Yasin Dhaher, Ph.D. (no lab name)
\medskip

\textbf{Graduate Research Assistant}
\hfill 2013 -- 2019\\ % 10/2013 -- 05/2019
University of Wisconsin–Madison, Dept. of Mechanical Engineering  % Madison, WI
% PI: Darryl Thelen, Ph.D. (Neuromuscular Biomechanics Laboratory)
\medskip

\textbf{Research Assistant}
\hfill 2010 -- 2012\\ % 08/2010 -- 12/2012
Brigham Young University, Dept. of Mechanical Engineering %Provo, UT
% PI: Steven Charles, Ph.D. (Neuromechanics Research Group)
\medskip

\textbf{Systems Engineer}
\hfill 2010 -- 2010\\ % 04/2010 -- 08/2010
The Boeing Company, Airborne Early Warning \& Control %, Kent, WA, Airborne Early Warning \& Control: Project Wedgetail ISS 

%---------------------------------------------------------------------------------------------------
\section{Publications}

%\subsection{Manuscripts in Review:}
%\begin{enumerate}
%    \item Engdahl SM, \textbf{Acuña SA}, Kaliki RR, Sikdar S (2022). Sonomyography for control of upper limb prostheses: current state and future directions. \textit{Prosthetics and Orthotics}.
%\end{enumerate}

\subsection{Peer-Reviewed Publications:}
\begin{enumerate}
    \item Engdahl SM, \textbf{Acuña SA}, Kaliki RR, Sikdar S (2022). Sonomyography for control of upper limb prostheses: current state and future directions. \textit{Prosthetics and Orthotics}. (Accepted.)
    \item Majdi JA, \textbf{Acuña SA}, Chitnis PV, Sikdar S (2022). Towards a wearable monitor of local muscle fatigue during electrical stimulation using tissue Doppler imaging. \textit{Wearable Technologies}, 3: E16. [\href{https://www.cambridge.org/core/journals/wearable-technologies/article/toward-a-wearable-monitor-of-local-muscle-fatigue-during-electrical-muscle-stimulation-using-tissue-doppler-imaging/4ADA49B3A98D245E222D5CCFDE1F180D}{Link}]
    \item Engdahl SM, \textbf{Acuña SA}, King EL, Bashatah A, S Sikdar (2022). First-in-human demonstration of functional task performance using a sonomyographic prosthesis. \textit{Frontiers in Bioengineering \& Biotechnology}, 10: 876836. [\href{https://www.frontiersin.org/articles/10.3389/fbioe.2022.876836/full}{Link}]
    \item \textbf{Acuña SA}, Tyler ME, Thelen DG (2022). Individuals with chronic mild to moderate traumatic brain injury exhibit decreased neuromuscular complexity during gait. \textit{Neurorehabilitation and Neural Repair}, 36(4-5): 317-327. [\href{https://journals.sagepub.com/doi/full/10.1177/15459683221081064}{Link}]
    \item Schroeder MJ, \textbf{Acuña SA}*, Krishnan C, Dhaher YY (2022). Can increased locomotor task complexity differentiate knee muscle forces after ACL-Reconstruction? \textit{Applied Biomechanics}, 38(2): 84-94. [\href{https://journals.humankinetics.com/view/journals/jab/38/2/article-p84.xml}{Link}] (*Acuña is considered a co-first author.)
    \item \textbf{Acuña SA}, Ebrahimi A, Pomeroy RL, Martin JA, Thelen DG (2019). Achilles tendon shear wave speed tracks the dynamic modulation of standing balance. \textit{Physiological Reports}, 7: e14298. [\href{https://physoc.onlinelibrary.wiley.com/doi/full/10.14814/phy2.14298}{Link}]
    \item \textbf{Acuña SA}, Zunker JD, Thelen DG (2019). The effects of sub-threshold vibratory noise on visuomotor entrainment during human walking and standing in a virtual reality environment. \textit{Human Movement Science}, 66: 587-599. [\href{https://www.sciencedirect.com/science/article/abs/pii/S0167945719300582}{Link}]
    \item \textbf{Acuña SA}, Francis CA, Franz JR, Thelen DG (2019). The effects of cognitive load and optical flow on antagonist leg muscle coactivation during walking for young and older adults. \textit{Electromyography and Kinesiology}, 44: 8-14. [\href{https://www.sciencedirect.com/science/article/abs/pii/S1050641118303122}{Link}]
    \item \textbf{Acuña SA}, Tyler ME, Danilov YP, Thelen DG (2018). Abnormal muscle activation patterns are associated with chronic gait deficits following traumatic brain injury. \textit{Gait \& Posture}, 62: 510-517. [\href{https://www.sciencedirect.com/science/article/abs/pii/S096663621830359X}{Link}] (Finalist for best paper award, GCMAS 2017.)
    \item \textbf{Acuña SA}, Smith DM, Robinson JM, Hawks JC, Starbuck P, King DL, Ridge ST, Charles SK (2014). Instrumented figure skating blade for measuring on-ice skating forces. \textit{Measurement Science and Technology}, 25(12): 125901. [\href{https://iopscience.iop.org/article/10.1088/0957-0233/25/12/125901}{Link}]
\end{enumerate}

%\subsection{Invited Commentaries:}
%\subsection{Letters to the Editor:}
%\subsection{Written, but not yet submitted:} % in preparation
%\begin{enumerate}
%     \item 
%\end{enumerate}

%\subsection{Patents/IP:}
%\begin{enumerate}
%     \item 
%\end{enumerate}

\subsection{Other:}
\begin{enumerate}
    \item \textbf{Acuña SA} (2019). Altered neuromuscular control of gait following traumatic brain injury and targeted neuromodulation to improve motor function. \textit{The University of Wisconsin -- Madison. ProQuest Dissertations Publishing}, 13882699. [\href{https://www.proquest.com/docview/2229834647}{Link}]
\end{enumerate}

%---------------------------------------------------------------------------------------------------
\section{Conference Abstracts}
\begin{enumerate}
    \item \textbf{Acuña SA}, Bashatah A, Sikdar S. How to implement wearable ultrasound for prosthetic hand control. Presentation at the Biomedical Engineering Society annual conference. Seattle, WA. October 2023. (Submitted.)
    \item Taghizadeh Z, Bashatah A, \textbf{Acuña SA}, Sikdar S. Wearable ultrasound system for controlling upper limb prosthetics. Presentation at the National Assembly of the American Orthotic \& Prosthetic Association. Indianapolis, IN. September 2023. (Accepted.)
    \item \textbf{Acuña SA}, Bashatah A, Sikdar S. How to implement wearable ultrasound for prosthetic hand control. Presentation at the American Society of Biomechanics annual meeting. Knoxville, TN. August 2023. (Accepted.)
     \item \textbf{Acuña SA}, Krotine M, Gibson G, Boser Q, Hebert J, Sikdar S. Reproducibility of gaze and movement assessment of upper limb function. Presentation at the American Society of Biomechanics annual meeting. Knoxville, TN. August 2023. (Accepted.)
    \item \textbf{Acuña SA}, Bashatah A, Sikdar S. How to implement wearable ultrasound for prosthetic hand control. Podium presentation at the east coast regional conference for the American Society of Biomechanics. Reading, PA. April 2023.
     \item \textbf{Acuña SA}, Krotine M, Gibson G, Boser Q, Hebert J, Sikdar S. Reproducibility of gaze and movement assessment of upper limb function. Poster presentation at the east coast regional conference for the American Society of Biomechanics. Reading, PA. April 2023.
    \item \textbf{Acuña SA}, Bashatah A, Sutherland RF, Kaliki RR, Sikdar S. A wearable ultrasound system for controlling an upper-limb prosthesis. Presentation at the 49th Academy Annual Meeting \& Scientific Symposium of the American Academy of Orthotists \& Prosthetists. Nashville, TN. March 2023. [\href{https://journals.lww.com/jpojournal/Citation/2023/04001/UPPER_LIMB_PROSTHESES.10.aspx}{Link}]
    \item Patwardhan S, \textbf{Acuña SA}*, Engdhal SM, Mukherjee B, Gladhill KA, Bashata A, Dhawan AS, Abreu R, Schofield JS, Joiner WM,  Sikdar S. Comparison of virtual end-point trajectories using sonomyography with trajectories derived from coordinated multi-joint movements. Poster presentation at the Military Health System Research Symposium. Kissimmee, FL. September 2022. (*Acuña was added after submission as the presenting author.)
    \item \textbf{Acuña, SA}*, Labbé DR, Dingwell JB, Jenkins E. Using Virtual Reality for Physical Rehabilitation. Symposium presentation at the North American Congress on Biomechanics. Ottawa, ON. August 2022. (*Acuña organized and chaired the symposium.)
    \item \textbf{Acuña SA}, Engdahl SM, King EL, Bashatah A, Sikdar S. Reliability of sonomyography for controlling prosthetic hand grasps. Poster presentation at the North American Congress on Biomechanics. Ottawa, ON. August 2022.
    \item King EL, Engdahl SM, \textbf{Acuña SA}, Bashatah A, Sikdar S. Continuous testing of sonomyography as a control paradigm for upper limb prostheses. Poster presentation at the North American Congress on Biomechanics. Ottawa, ON. August 2022.
    \item \textbf{Acuña SA}, Engdahl SM, Bashatah A, Otto P, Kaliki RR, Sikdar S. A wearable sonomyography system for prosthesis control. Poster presentation at the Myoelectric Controls Symposium. Hosted by the Institute of Biomedical Engineering at the University of New Brunswick. Fredericton, NB. August 2022.
    \item Engdahl SM, \textbf{Acuña SA}, Bashatah A, Dhawan AS, King EL, Mukherjee B, Holley RJ, Monroe BJ, Lévay G, Kaliki RR, Sikdar S. Assessing the feasibility of using sonomyography for upper limb prosthesis control. Podium presentation at the Myoelectric Controls Symposium. Hosted by the Institute of Biomedical Engineering at the University of New Brunswick. Fredericton, NB. August 2022.
    \item Bashatah A, Rima AH, King EL, Kaur A, \textbf{Acuña SA}, Chitnis PV, Sikdar S. Wearable ultrasound for rehabilitation applications. Presentation at the 46th International Symposium on Ultrasonic Imaging and Tissue Characterization. Virtual Meeting. June 2022.
    \item Engdahl SM, Mukherjee B, Dhawan AS, Bashatah A, Patwardhan S, \textbf{Acuña SA}, King EL, Lancaster BC, Akhlaghi N, Holley RJ, Monroe BJ, Kaliki RR, Sikdar S. Development of a novel ultrasound-based modality for control of upper limb prostheses. Presentation at the Trent International Prosthetics Symposium. Hosted online by the International Society for Prosthetics and Orthotics member societies of the United Kingdom and Netherlands. March 2022. (Winner of the Best Paper Award.)
    \item \textbf{Acuña SA}, Bashatah A, Chitnis PV, Sikdar S. Measuring signal quality in low power wearable ultrasound imaging. Presentation at the Acoustical Society of America. Seattle, WA. November 2021. [\href{https://asa.scitation.org/doi/10.1121/10.0007727}{Link}]
    \item \textbf{Acuña SA}. Quantitative electromyographic analysis can inform treatment planning for gait disorders. Presentation at the Frontiers of Computing in Health and Society, Institute for Digital Innovation, George Mason University. Fairfax, VA. September 2021.
    \item \textbf{Acuña SA}, Schroeder MJ, Krishnan C, Dhaher YY. Increased task demand differentiates knee muscle forces after ACL-reconstruction. Presentation at the American Society of Biomechanics Annual Meeting. Atlanta, GA. August 2020.
    \item \textbf{Acuña SA}, Kunnappally JR, Soedirdjo SDH, Phan P, Kim H, Rodriguez LA, Hutcherson CW, Chung YC, Dhaher YY. The role of estrogen on reciprocal inhibition of the Soleus. Oral presentation at the XXIII Congress of the International Society of Electrophysiology and Kinesiology. Nagoya, Japan. July 2020.
    \item Soedirdjo SDH, \textbf{Acuña SA}, Kunnappally JR, Phan P, Kim H, Rodriguez LA, Hutcherson CW, Chung YC, Dhaher YY. Isolated mixed effect of estradiol and progesterone on motor neuron excitability. Oral presentation at the XXIII Congress of the International Society of Electrophysiology and Kinesiology. Nagoya, Japan. July 2020.
    \item \textbf{Acuña SA}, Ebrahimi A, Thelen DG. Achilles tendon shear wave speed tracks the dynamic modulation of standing balance. Oral presentation at the XXIII Congress of the International Society of Electrophysiology and Kinesiology. Nagoya, Japan. July 2020.
    \item \textbf{Acuña SA}, Dhaher YY. Individuals with chronic traumatic brain injury exhibit decreased neuromuscular complexity when walking: an overview of neuromechanics research. Podium presentation at the UT Southwestern Postdoctoral Association Annual Research Symposium. Dallas, TX. September 2019.
    \item \textbf{Acuña SA}, Ebrahimi A, Thelen DG. Achilles tendon shear wave speed as a measure of the active modulation of standing balance. Podium and poster presentation at the joint conference of the International Society of Biomechanics and American Society of Biomechanics. Calgary, AB. August 2019. (Finalist for ASB Doctoral Student Presentation Competition.)
    \item \textbf{Acuña SA}, Zunker JD, Thelen DG. Sub-threshold vibratory noise does not alter visuomotor entrainment during human walking. Poster presentation at the Gait and Clinical Motion Analysis Society Annual Meeting. Frisco, TX. March 2019.
    \item \textbf{Acuña SA}, Tyler ME, Danilov YP, Thelen DG. Changes in dynamic motor control following neurorehabilitation for traumatic brain injury: treadmill vs overground walking. Podium and poster presentation at the American Society of Biomechanics Annual Meeting. Rochester, MN. August 2018. (Finalist for ASB Doctoral Student Presentation Competition.)
    \item \textbf{Acuña SA}, Tyler ME, Danilov YP, Thelen DG. Improvements in dynamic motor control following neurorehabilitation of chronic balance deficits due to prior traumatic brain injury. Podium presentation at the 8th World Congress of Biomechanics. Dublin, Ireland. July 2018. (Runner up for the ASME-BED PhD Level Student Paper Competition.)
    \item \textbf{Acuña SA}, Francis CA, Franz JR, Thelen DG. Walking with visual perturbations but not an attention-dividing task modulates muscle coactivation patterns in old adults. Podium presentation at the XXII Congress of the International Society of Electrophysiology and Kinesiology. Dublin, Ireland. June 2018.
    \item \textbf{Acuña SA}, Michaelis JE, Roth JD, Towles JD. Intervention designed to increase interest in engineering for low-interest, K-12 girls did so for boys and girls. Presentation at the American Society for Engineering Educations Annual Conference and Exposition. Salt Lake City, UT. June 2018.
    \item \textbf{Acuña SA}, Tyler ME, Danilov YP, Thelen DG. Effect of non-invasive neuromodulation on rehabilitation of gait in chronic traumatic brain injury. Podium presentation at the Gait and Clinical Motion Analysis Society Annual Meeting. Indianapolis, IN. May 2018.
    \item \textbf{Acuña SA}, Tyler M, Danilov Y, Thelen DG. Individuals with a prior traumatic brain injury exhibit decreased neuromuscular complexity during gait. Thematic poster presentation at the American Society of Biomechanics Annual Meeting. Boulder, CO. August 2017.
    \item Zunker JD, \textbf{Acuña SA}, Thelen DG. Piezoelectric device for peripheral stochastic sub sensory vibration. Poster presentation at the American Society of Biomechanics Annual Meeting. Boulder, CO. August 2017.
    \item Francis CA, Michaelis JE, \textbf{Acuña SA}, Towles JD. Impact of Biomechanics-based activities on situational and individual interest among K-12 students. Podium presentation at the 2017 American Society for Engineering Education Annual Conference and Exposition. Columbus, OH. June 2017.
    \item \textbf{Acuña SA}, Tyler M, Danilov Y, Thelen DG. Muscle activation patterns during walking are correlated to clinical gait assessments after traumatic brain injury. Podium presentation at the Gait and Clinical Movement Analysis Society Annual Meeting. Salt Lake City, UT. May 2017. (Nominated for best paper.)
    \item \textbf{Acuña SA}, Thelen DG. Cranial nerve non-invasive neuromodulation for symptomatic treatment of traumatic brain injury. Poster presentation at the Opportunities in Engineering Annual Conference. Madison, WI. November 2016.
    \item Francis CA, Franz JR, \textbf{Acuña SA}, Thelen DG. Gait and balance training improves gait variability in older adults. Thematic poster presentation at the American Society of Biomechanics Annual Meeting. Raleigh, NC. August 2016.
    \item \textbf{Acuña SA}, Tyler M, Danilov Y, Thelen DG. Cranial nerve non-invasive neuromodulation for symptomatic treatment of mild and moderate traumatic brain injury: effects on muscle coordination patterns during walking. Podium presentation at the XXI Congress of the International Society of Electrophysiology and Kinesiology. Chicago, IL. July 2016.
    \item \textbf{Acuña SA}, Tyler M, Danilov Y, Thelen DG. Cranial nerve non-invasive neuromodulation for symptomatic treatment of mild and moderate traumatic brain injury: effects on muscle coordination patterns during walking. Poster presentation at the Dynamic Walking Conference: Principles of Dynamic Locomotion. Holly, MI. June 2016.
    \item \textbf{Acuña SA}, Thelen DG. Efforts for preventing falls in the elderly via stochastic resonance. Poster presentation at the Opportunities in Engineering Annual Conference. Madison, WI. October 2015.
    \item \textbf{Acuña SA}, Towles JD, Thelen DG. Modeling based analysis of the trapezial-metacarpal joint to reduce osteoarthritis. Poster presentation at the Opportunities in Engineering Annual Conference. Madison, WI. November 2014.
    \item Smith DM, \textbf{Acuña SA}, Hawks JC, Packard JG, Robinson JM, King DL, Ridge ST, Charles SK. System for measuring figure skate forces on ice. Poster presentation at the 7th World Congress of Biomechanics. Boston, MA. July 2014.
    
\end{enumerate}

%---------------------------------------------------------------------------------------------------
\section{Invited Presentations}
\begin{enumerate}
    \item  Wearable ultrasound for controlling prosthetic hands. Department of Mechanical Engineering, Brigham Young University. Provo, UT. May 2022.
    \item  Research and teaching in biomedical engineering education, University of the District of Columbia. Washington, DC. May 2022.
    \item  Improving physical rehabilitation with virtual reality and wearable ultrasound. Department of Mechanical Engineering, Gonzaga University. Spokane, WA. April 2022.
    \item  Towards wearable ultrasound imaging with the Achilles tendon. Medical Engineering Conference, hosted by the Biomedical Engineering Society at George Mason University. Fairfax, VA. March 2022.
    \item  Using ultrasound to measure Achilles tendon kinematics, kinetics, and material properties when walking. Acoustical Society of America Annual Conference. Seattle, WA. November 2021. [\href{https://asa.scitation.org/doi/10.1121/10.0008319}{Link}]
    \item  Improvements in dynamic motor control following neurorehabilitation of traumatic brain injury. Biomedical Engineering Guest Lecture Series. University of the District of Columbia. Washington, DC. January 2021.
    \item  Becoming successful product design engineers. Future Faculty Career Exploration Program. Rochester Institute of Technology. Rochester, NY. September 2018.
    \item  Non-invasive neuromodulation to improve upright balance when walking. Neuromechanics seminar. Brigham Young University. Provo, UT. May 2017.
    \item  Maintenance of balance with aging: choose your steps carefully. 28th Annual Colloquium on Aging. UW--Madison Institute on Aging. Madison, WI. September 2016. (\textit{Voted most popular speaker by colloquium attendees.})
    \item  Maintaining balance while aging: choose your steps carefully. The Wisconsin Institutes for Discovery: Noon @ the Niche lecture series. University of Wisconsin--Madison. Madison, WI. March 2016.
    \item Maintenance of balance with aging: choose your steps carefully. UW--Madison Institute on Aging Materials Science Program. Madison, WI. October 2015.
\end{enumerate}

%---------------------------------------------------------------------------------------------------
\section{Grant Support}

%\subsection{Current:}
%\begin{itemize}
%    \item[] 
%\end{itemize}


\subsection{Completed:}
\begin{itemize}
    \item[] R01HD092697-01S1 (PI: Thelen DG) \hfill 2018 -- 2019\\ % 03/2018 -- 12/2019
    NIH Eunice Kennedy Schriver National Institute of Child Health \& Human Development\\
    Research Supplement to Promote Diversity in Health-Related Research\\
    “Noninvasive assessment of in vivo tissue loads to enhance the treatment of gait disorders”\\
    Role: Co-Investigator\\
    Amount: \$35,915\\
    
    \item[] R25GM083252 (PI: Carnes, ML) \hfill 2015 -- 2017\\
    NIH General Medical Sciences\\
    Initiative for Maximizing Student Development\\
    "Training and education to advance minority scholars in science (TEAM-Science)"\\
    Role: Trainee\\
    
    \item[] Graduate Research Scholar Fellowship \hfill 2014 -- 2018\\
    State of Wisconsin Advanced Opportunity Program \& Wisconsin Alumni Research Foundation\\
    Community: Graduate Engineering Research Scholars, University of Wisconsin--Madison\\

\end{itemize}

%---------------------------------------------------------------------------------------------------
\section{Honors \& Awards}

Finalist, International "Rethink EMG" Competition. Delsys Inc. \hfill 2022\\
Best paper award (with Engdahl SM). Trent International Prosthetics Symposium. \hfill 2022\\
Finalist, Graduate Student Rapid Poster Award Competition, International Society of Biomechanics. \hfill 2019\\
Travel Award, Education Council of the Gait and Clinical Movement Analysis Society. \hfill 2019\\
3rd place, Engineering Expo Graduate Exhibits, UW--Madison. \hfill 2019\\
Finalist, Doctoral Student Presentation Competition, American Society of Biomechanics. \hfill 2018\\
Runner Up, ASME-BED PhD Level Student Paper Competition, 8th World Congress of Biomechanics. \hfill 2018\\
Student Travel Grant, De Luca Foundation, 8th World Congress of Biomechanics. \hfill 2018\\
1st place, Engineering Expo Graduate Exhibits, UW--Madison. \hfill 2018\\
Kevin Granata Young Investigator Award, Gait and Clinical Movement Analysis Society. \hfill 2017\\
Finalist, Best Paper Award, Gait and Clinical Movement Analysis Society. \hfill 2017\\
Student Travel Grant, De Luca Foundation, American Society of Biomechanics. \hfill 2017\\
Greatest Impact Award, National Biomechanics Day Student Competition. \hfill 2017\\
1st place, Engineering Expo Graduate Exhibits, UW--Madison. \hfill 2017\\
Travel Award, Education Council of the Gait and Clinical Movement Analysis Society. \hfill 2017\\
Mechanical Engineering---Graduate School Physical Sciences Division Fellowship, UW--Madison. \hfill 2016\\
1st place, Engineering Expo Graduate Exhibits, UW--Madison. \hfill 2016\\
Diversity Travel Award, American Society of Biomechanics. \hfill 2015\\
NCEES Fundamentals of Engineering Exam. \hfill 2012\\


%---------------------------------------------------------------------------------------------------
\section{Professional Organizations}

\begin{compacthang}
     \item Member, Center for Adaptive Systems of Brain-Body Interactions, George Mason University
     \item Member, American Society of Biomechanics
     \item Member, Biomedical Engineering Society
     \item Member, International Society of Electrophysiology and Kinesiology
     \item Member, Gait and Clinical Movement Analysis Society
     \item Affiliate Member, UW--Madison Teaching Academy 
     \item Affiliate Member, National Postdoctoral Association
\end{compacthang}

%---------------------------------------------------------------------------------------------------
\section{Professional Service}

\subsection{Manuscript Review:}
\setlength{\hangingleftmargin}{2.5em}
\setlength{\hangingindent}{2em}
\begin{compacthang}
    \item Biomechanics
    \item Gait \& Posture
    \item Sensors
    \item Ultrasonic Imaging
\end{compacthang}

\subsection{Conference Paper Review:}
\begin{hanginglist} %\begin{compacthang}
    \item American Society of Biomechanics
    \item American Society of Mechanical Engineers, International Design Engineering Technical Conferences \& Computers and Information in Engineering Conference.
\end{hanginglist} %\end{compacthang}

\subsection{Grant Review:}
\begin{hanginglist}
    \item National Science Foundation reviewer (2023) % they request not to list GRFP on the CV. Graduate Research Fellowship Program (GRFP)
    \item Special Projects in Rehabilitation Excellence( SPiRE), Veterans Health Administration Office of Research and Development.
    \item American Orthotic \& Prosthetic Association (AOPA), Center for Orthotic \& Prosthetic Learning and Outcomes/Evidence-Based Practice (COPL)
\end{hanginglist}

\subsection{Organized Symposiums:}
\begin{hanginglist}
     \item Using Virtual Reality for Physical Rehabilitation. North American Congress on Biomechanics. Ottawa, ON. August 2022.
\end{hanginglist}

\subsection{Conference Organization Committees:}
\begin{hanginglist}
     \item East coast regional conference, American Society of Biomechanics. \hfill 2022 -- 2023
\end{hanginglist}


\subsection{Key Roles:}
\begin{itemize}
%    \item Conference Chair, Regional conference for the American Society of Biomechanics, George Mason University. 2022-23.  
    \item[] Faculty Advisor, GMU Student Chapter -- American Society of Biomechanics. \hfill 2023 -- Present
    \item[] Student Advisory Committee, American Society of Biomechanics. \hfill 2016 -- 2019 %09/2016 – 08/2019
    \item[] Chair, National Biomechanics Day Committee. University of Wisconsin--Madison. \hfill 2015 -- 2018
\end{itemize}

\subsection{STEM Outreach:}
\begin{itemize}
    \item[] National Biomechanics Day presentation. College Readiness Early Identification Program, GMU. \hfill 2023
    \item[] National Biomechanics Day presentation. College Readiness Early Identification Program, GMU. \hfill 2022 %March 2022. Intro to Bioengineering STEM Fusion class
    \item[] Career Day Panelist. College Readiness Early Identification Program, GMU. \hfill 2021 %September 2021.
    \item[] National Biomechanics Day, Engineering Expo exhibit. UW--Madison. \hfill 2019 %April 2019. : The Human Machine
    \item[] National Biomechanics Day, Engineering Expo exhibit. UW--Madison. \hfill 2018 %April 2018. : The Human Machine
    \item[] After school activity. Nuestro Mundo Community School. Monona, WI. \hfill 2018
    \item[] Presentation. Teen Science Cafe. Wisconsin Institute for Discovery. Madison, WI. \hfill 2017 %April 2017. Presentation on Electromyography
    \item[] National Biomechanics Day, Engineering Expo exhibit. UW--Madison. \hfill 2017 % April 2017. : The Human Machine
    \item[] After school activity. Nuestro Mundo Community School. Monona, WI. \hfill 2017
    \item[] Classroom presentation. Federal Way High School. Federal Way, WA. \hfill 2016 %September 2016. Presentation on Product Design
    \item[] National Biomechanics Day, Engineering Expo exhibit. UW--Madison. \hfill 2016 %April 2016. : The Human Machine
    \item[] After school activity. Nuestro Mundo Community School. Monona, WI. \hfill 2016
    \item[] National Biomechanics Day, Engineering Expo exhibit. UW--Madison. \hfill 2015 %April 2015. : Engineering the Super Human
    \item[] After school activity. Nuestro Mundo Community School. Monona, WI. \hfill 2015
    \item[] Classroom presentation. Akira Toki Middle School. Madison, WI. \hfill 2014 % October 2014. Presentation on Electromyography
\end{itemize}

\subsection{Other:}
\begin{itemize}
    \item[] Senior Design Adjudicator. Jonsson School of Engineering \& Computer Science, UT Dallas. \hfill 2019 %December 2019. Senior Design Capstone Program
\end{itemize}

%---------------------------------------------------------------------------------------------------
\section{Teaching Experience}

\textbf{George Mason University}\\
PROV 801/802 (Community-Engaged Interdisciplinary Methods), Instructor \hfill 2023 -- Present\\
BENG 501 (Research Methods), Instructor \hfill 2022 -- Present\\
BENG 492/493 (Senior Advanced Design), Faculty Advisor for Senior Design Team \hfill 2021 -- Present\\
BENG 375/575 (Intellectual Prop., Regulatory Concepts, and Product Dev.), Instructor \hfill 2022\\
PROV 801/802 (Community-Engaged Interdisciplinary Methods), Guest Lecturer \hfill 2022 -- 2023\\
ME 443/444 (Mechanical Design), Faculty Advisor for Senior Design Team \hfill 2021 -- 2022\\
BENG 391 (Professional Development), Guest Lecturer \hfill 2021 -- 2022\\
%\textbf{Fall 2022}, Instructor, BENG 501 (Research Methods)\\
%\textbf{Fall 2022}, Instructor, BENG 375/575 (Intellectual Property, Regulatory Concepts, and Product Development)\\
%\textbf{Fall 2022}, Guest Lecture, BENG 391 (Professional Development)\\
%\textbf{Fall 2022 -- Spring 2023}, Faculty Advisor for Senior Design Team, BENG 492/493 (Senior Advanced Design)\\ 
%\textbf{Fall 2021 -- Spring 2022}, Faculty Advisor for Senior Design Team, BENG 492/493 (Senior Advanced Design)\\ 
%\textbf{Fall 2021}, Guest Lecture, BENG 391 (Professional Development)
%\textbf{Summer 2022}, Guest Lecture, PROV801/802 (Community-Engaged Interdisciplinary Methods)\\
%\textbf{Spring 2022}, Guest Lecture, PROV801/802 (Community-Engaged Interdisciplinary Methods)
%\textbf{Fall 2021 -- Spring 2022}, Faculty Advisor for Senior Design Team, ME 443/444 (Mechanical Design)

\textbf{Gonzaga University}\\
MENG 330 (Machine Design), Guest Lecture \hfill 2022\\
%\textbf{Spring 2022}, Guest Lecture, MENG 330 (Machine Design)

\textbf{University of Texas at Dallas}\\
MECH 4382 (Senior Design), Technical Manager \hfill 2020\\
%\textbf{Spring 2020}, Technical Manager, MECH 4382 (Senior Design)

\textbf{University of Wisconsin--Madison}\\
ME 549 (Product Design), Teaching Assistant and Lecturer \hfill 2015 -- 2018\\
BME 200/201/300/301 (Biomedical Engineering Design), Mentor for Student Design Team \hfill 2016 -- 2017\\
Pre-College Enrichment Opportunity Program for Learning Excellence (Mechatronics), Instructor \hfill 2017\\
%\textbf{Fall 2018}, Teaching Assistant and Co-Instructor, ME 549 (Product Design)\\
%\textbf{Fall 2017}, Teaching Assistant and Co-Instructor, ME 549 (Product Design)\\ 
%\textbf{Fall 2016}, Teaching Assistant and Co-Instructor, ME 549 (Product Design)\\ 
%\textbf{Fall 2015}, Teaching Assistant, ME 549 (Product Design)
%\textbf{Spring 2017}, Mentor for Student Design Team, BME 201/301 (Biomedical Engineering Design)\\
%\textbf{Fall 2016}, Mentor for Student Design Team, BME 200/300 (Biomedical Engineering Design)
%\textit{University of Wisconsin--Madison, Pre-College Enrichment Opportunity Program for Learning Excellence}\\
%\textbf{Summer 2017}, Instructor, Engineering Workshop (Mechatronics for Product Design)

\textbf{University of Wisconsin--Milwaukee}\\
KINES 910 (Advanced Seminar in Health Sciences), Guest Lecture \hfill 2016\\
%\textbf{Spring 2016}, Guest Lecture, KINES 910 (Advanced Seminar in Health Sciences)

\textbf{Brigham Young University}\\
ME 373 (Scientific Computing and Computer Aided Engineering), Teaching Assistant \hfill 2012\\
%\textbf{Fall 2012}, Teaching Assistant, ME 373 (Scientific Computing and Computer Aided Engineering)


%---------------------------------------------------------------------------------------------------
\section{Mentoring Experience}
\textbf{Graduate Students}\\
Erica King, George Mason University (Bioengineering) \hfill 2021 -- Present\\
Joseph Majdi, George Mason University (Bioengineering) \hfill 2021 -- 2022\\
Tony Kim, University of Texas--Dallas (Biomedical Engineering) \hfill 2019 -- 2020\\

\textbf{Undergraduate Students}\\
Gabriel Gibson, George Mason University (Bioengineering) \hfill 2022 -- Present\\
Maddie Krotine, University of Virginia (Biomedical Engineering) \hfill 2022\\
Ryan Devlin, University of Texas--Dallas (Biomedical Engineering) \hfill 2020\\
Bailey Ramesh, UW--Madison (Biomedical Engineering) \hfill 2017 -- 2018\\
Isaac Loegering, UW--Madison (Biomedical Engineering) \hfill 2016\\
John Zunker, UW--Madison (Mechanical Engineering) \hfill 2015 -- 2018\\

\textbf{Mentored Student Honors and Awards}\\
Faustin Prinz Undergraduate Research Fellowship, John Zunker \hfill 2016

%---------------------------------------------------------------------------------------------------
\section{Professional Development}
Proposal Writing (ME 699). 15-week course. George Mason University. \hfill 2022\\
Preparing Future Faculty Workshop. 4-week course. Auburn University. \hfill 2021\\ 
Responsible Conduct of Research. 9-week course. UT Southwestern. \hfill 2020\\
Preparation for a Scientific Career. 9-week course. UT Southwestern. \hfill 2019\\
Information Mastery for Postdoctoral Trainees. 9-week course. UT Southwestern. \hfill 2019\\
Future Faculty Career Exploration Program. 1-week course. Rochester Institute of Technology. \hfill 2018\\ 
Research Mentor Training. 14-week course. UW--Madison. \hfill 2018\\
Teaching in Science and Engineering. 14-week course. UW--Madison. \hfill 2017\\
Effective Teaching with Technology. 14-week course. UW--Madison. \hfill 2016\\
Improv to improve Teaching \& Science Communication. 14-week course. UW--Madison. \hfill 2016\\

%---------------------------------------------------------------------------------------------------
% SKILLS (optional section)
%transferrable skills:
%Research, Analysis, Interpretation & Reporting
%Team Leadership, Training, Supervision
%Content Development & Instruction
%Scientific Writing & Public Speaking
%Mentoring and Teaching
%Ability to Translate Complex Concepts
%
%laboratory skills:
%Human Subjects Testing
%Electromyography 
%Biological Systems Signal Processing
%Mechanical Prototyping
%Computer Programming (Matlab, LabView, C++)
%Gait Analysis & Motion Capture
%Systems Neurophysiology Testing
%Musculoskeletal Modeling & Simulation
%Functional Magnetic Resonance Imaging
%Bioinstrumentation Design
%
%areas of expertise:
%Mechanical Engineering
%Neuromuscular Biomechanics
%Rehabilitation Engineering
%Non-Invasive Neuromodulation
%Biofeedback & Signal Processing
%Patent Procurement
%Biomedical Engineering
%Human-Centered Product Design
%Movement Disorder, esp altered Gait & Posture
%Biomechanics of Aging
%Muscle Activation & Motor Control
%Intellectual Property




\end{document}